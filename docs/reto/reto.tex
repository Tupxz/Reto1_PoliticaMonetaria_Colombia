% =============================================================================
% RETO 1: POLÍTICA MONETARIA Y INFLACIÓN EN COLOMBIA
% Análisis ADL(2,2,2,2) - Econometría II EAFIT
% =============================================================================

\documentclass[12pt, a4paper, spanish]{article}
\usepackage[spanish]{babel}
\usepackage[utf8]{inputenc}
\usepackage{amsmath}
\usepackage{amssymb}
\usepackage{graphicx}
\usepackage{booktabs}
\usepackage{hyperref}
\usepackage{xcolor}
\usepackage{listings}
\usepackage{geometry}
\geometry{margin=1in}

% Colores para hipervínculos
\hypersetup{colorlinks=true, linkcolor=blue, urlcolor=blue}

% Configuración de código
\lstset{
    language=R,
    basicstyle=\ttfamily\small,
    keywordstyle=\color{blue},
    commentstyle=\color{gray},
    stringstyle=\color{red},
    breaklines=true,
    showstringspaces=false
}

% =============================================================================
% DOCUMENTO
% =============================================================================

\title{\textbf{Análisis de Política Monetaria y Transmisión de Inflación \\
en Colombia (2006-2026): \\
Un Modelo Autoregresivo con Rezagos Distribuidos}}

\author{
  Econometría II - EAFIT \\
  Reto 1: Haciendo Macroeconomía
}

\date{Febrero 16, 2026}

\begin{document}

\maketitle

% =============================================================================
\section*{Resumen Ejecutivo}
% =============================================================================

Este documento presenta un análisis econométrico riguroso del efecto de la 
política monetaria sobre la inflación en Colombia durante el período enero 
2006 - noviembre 2025. Utilizamos un modelo Autoregresivo con Rezagos 
Distribuidos (ADL) de especificación ADL(2,2,2,2) estimado por Mínimos 
Cuadrados Ordinarios.

\textbf{Hallazgo principal:} La política monetaria (medida por la tasa DTF) 
\textit{es efectiva} para controlar la inflación. Un aumento de 100 puntos 
base en la DTF reduce la inflación en aproximadamente \textbf{23.07 puntos 
base en corto plazo} y \textbf{32.79 puntos base en largo plazo}, con 
significancia estadística al nivel 1\% (p = 0.0025).

\bigskip

\noindent
\textbf{Palabras clave:} Política Monetaria, Inflación, ADL, Transmisión 
Monetaria, Colombia, Economía Abierta.

% =============================================================================
\section{Introducción}
% =============================================================================

La efectividad de la política monetaria es una de las preguntas fundamentales 
de la macroeconomía. En el caso de Colombia, el Banco de la República ha 
adoptado un régimen de inflación objetivo desde 2006, con una meta central de 
3\% anual (rango \(\pm 1\%\)).

La pregunta que abordamos en este análisis es: \textit{¿Realmente funciona la 
política monetaria en Colombia?}

Para responderla, estimamos un modelo econométrico que relaciona cambios en la 
tasa de interés oficial (DTF) con la inflación anual, controlando por otros 
determinantes importantes como el tipo de cambio (TRM) y la actividad económica 
(ISE).

% =============================================================================
\section{Marco Teórico}
% =============================================================================

\subsection{Canales de Transmisión de la Política Monetaria}

La literatura económica identifica varios canales por los cuales un aumento en 
la tasa de interés de política monetaria reduce la inflación:

\begin{enumerate}
    \item \textbf{Canal de Demanda Agregada:} Tasas más altas 
          \(\rightarrow\) menor inversión privada y consumo 
          \(\rightarrow\) menor presión inflacionaria.
    
    \item \textbf{Canal de Expectativas:} Credibilidad en la autoridad 
          monetaria \(\rightarrow\) inflación esperada menor 
          \(\rightarrow\) inflación actual menor (via Curva de Phillips).
    
    \item \textbf{Canal de Pass-Through Cambiario:} Tasas más altas 
          \(\rightarrow\) entrada de capitales \(\rightarrow\) 
          apreciación del tipo de cambio \(\rightarrow\) importables más 
          baratos \(\rightarrow\) inflación menor.
    
    \item \textbf{Canal de Cartera:} Tasas más altas \(\rightarrow\) 
          sustitución hacia activos nacionales \(\rightarrow\) menor 
          presión depreciacionista.
\end{enumerate}

\subsection{Especificación ADL}

Utilizamos un modelo Autoregresivo con Rezagos Distribuidos (ADL) como forma 
reducida para capturar estas dinámicas:

\begin{equation}
\begin{split}
y_t &= \alpha + \sum_{i=1}^{p} \phi_i y_{t-i} + \sum_{i=0}^{q_1} \beta_i x_{1,t-i} \\
    &\quad + \sum_{i=0}^{q_2} \gamma_i x_{2,t-i} + \sum_{i=0}^{q_3} \delta_i x_{3,t-i} + \epsilon_t
\end{split}
\end{equation}

donde:
\begin{itemize}
    \item $y_t = \Delta_{12} \log(\text{IPC}_t)$ es la inflación anual
    \item $x_{1,t}$ es la DTF en niveles (política monetaria)
    \item $x_{2,t}$ es $\Delta_{12} \log(\text{TRM}_t)$ (depreciación)
    \item $x_{3,t}$ es $\log(\text{ISE}_t)$ (actividad económica)
    \item $p$ es el orden de rezagos AR
    \item $q_j$ son los órdenes de rezagos distribuidos
\end{itemize}

% =============================================================================
\section{Datos y Variables}
% =============================================================================

\subsection{Período y Frecuencia}

\begin{itemize}
    \item \textbf{Período:} Enero 2006 - Noviembre 2025 (227 observaciones 
          mensuales después de transformaciones)
    \item \textbf{Frecuencia:} Mensual
    \item \textbf{Justificación:} Cubre múltiples ciclos económicos y el 
          régimen de inflación objetivo del BR
\end{itemize}

\subsection{Variable Dependiente: Inflación Anual}

\[
\text{inflación\_anual}_t = \Delta_{12} \log(\text{IPC}_t) = 
\log(\text{IPC}_t) - \log(\text{IPC}_{t-12})
\]

\textbf{Estadísticas:}
\begin{itemize}
    \item Media: 4.79\%
    \item Desviación Estándar: 2.58\%
    \item Rango: [1.48\%, 12.52\%]
\end{itemize}

\textbf{Justificación de transformación:}
\begin{enumerate}
    \item Elimina estacionalidad del IPC
    \item Comparable con meta del BR (3\% anual)
    \item Reduce volatilidad y mejora estacionariedad
\end{enumerate}

\subsection{Variable de Política Monetaria: DTF}

\[
\text{dtf\_nivel}_t = \text{Tasa promedio mensual DTF} \quad (\%)
\]

\textbf{Estadísticas:}
\begin{itemize}
    \item Media: 6.21\%
    \item Desviación Estándar: 2.95\%
    \item Rango: [1.76\%, 14.39\%]
\end{itemize}

\textbf{Justificación de niveles (no diferenciada):}
DTF es típicamente I(1) pero posiblemente cointegrada con inflación. El modelo 
ADL con componente AR captura la relación de equilibrio a largo plazo.

\subsection{Variable de Pass-Through: Depreciación Anual}

\[
\Delta_{12} \log(\text{TRM}_t) = \log(\text{TRM}_t) - \log(\text{TRM}_{t-12})
\]

\textbf{Interpretación:} Depreciación anual esperada del peso.

\textbf{Estadísticas:}
\begin{itemize}
    \item Media: 2.93\%
    \item Desviación Estándar: 13.30\%
    \item Rango: [-27.88\%, 46.50\%]
\end{itemize}

\subsection{Variable de Actividad Económica: ISE}

\[
\text{ise\_log}_t = \log(\text{ISE\_DAE\_desestacionalizado}_t)
\]

\textbf{Justificación:}
\begin{itemize}
    \item ISE desestacionalizado: coherencia con inflación anual
    \item ISE total (9 actividades): representatividad completa vs. subsector
    \item Proxy para presión de demanda agregada
\end{itemize}

% =============================================================================
\section{Metodología Econométrica}
% =============================================================================

\subsection{Especificación ADL(p,q)}

Evaluamos una grilla de 24 especificaciones:
\begin{itemize}
    \item $p \in \{1, 2, 3\}$ para rezagos AR
    \item $q_1, q_2, q_3 \in \{1, 2\}$ para rezagos distribuidos
\end{itemize}

\textbf{Selección de modelo:} Criterio AIC (Akaike Information Criterion).

\textbf{Modelo seleccionado:} ADL(2,2,2,2) con AIC = 12.25

\subsection{Estimación por OLS}

El modelo se estima mediante Mínimos Cuadrados Ordinarios usando el paquete 
\texttt{dynlm} de R (dynamic linear models).

\textbf{Supuestos clásicos:}
\begin{enumerate}
    \item [OK] Linealidad en parámetros
    \item [ALERTA] Autocorrelación residual (Ljung-Box p < 0.05 en lags altos)
    \item [OK] Homocedasticidad (Breusch-Pagan p = 0.056)
    \item [OK] Normalidad (Shapiro-Wilk p = 0.151)
\end{enumerate}

\subsection{Multiplicadores}

\textbf{Multiplicador de Corto Plazo:}
\[
MP_{CP} = \beta_0
\]

\textbf{Multiplicador de Largo Plazo:}
\[
MP_{LP} = \frac{\sum_{i=0}^{q} \beta_i}{1 - \sum_{i=1}^{p} \phi_i}
\]

donde el denominador es el factor de amplificación debido a la persistencia 
inflacionaria.

% =============================================================================
\section{Resultados}
% =============================================================================

\subsection{Estimación del Modelo ADL(2,2,2,2)}

\begin{table}[h]
\centering
\caption{Coeficientes Estimados del Modelo ADL(2,2,2,2)}
\label{tab:coef}
\begin{tabular}{lrrrr}
\toprule
Variable & Coeficiente & SE & t-stat & p-valor \\
\midrule
(Intercept) & 0.0429 & 0.0458 & 0.937 & 0.350 \\
$L(\text{inflación}, 1)$ & 1.3583 & 0.0623 & 21.798 & $<$0.001 *** \\
$L(\text{inflación}, 2)$ & -0.3958 & 0.0617 & -6.415 & $<$0.001 *** \\
$\text{DTF}(t)$ & 0.2307 & 0.0754 & 3.061 & 0.003 ** \\
$L(\text{DTF}, 1)$ & -0.0925 & 0.1269 & -0.729 & 0.467 \\
$L(\text{DTF}, 2)$ & -0.1260 & 0.0719 & -1.751 & 0.081 . \\
$\Delta_{12}\log(\text{TRM})(t)$ & 0.2065 & 0.3676 & 0.562 & 0.575 \\
$L(\Delta_{12}\log(\text{TRM}), 1)$ & 1.0524 & 0.5763 & 1.826 & 0.069 . \\
$L(\Delta_{12}\log(\text{TRM}), 2)$ & -0.8435 & 0.3880 & -2.174 & 0.031 * \\
$\log(\text{ISE})(t)$ & -0.6342 & 0.7005 & -0.905 & 0.366 \\
$L(\log(\text{ISE}), 1)$ & 3.7858 & 0.9091 & 4.164 & $<$0.001 *** \\
$L(\log(\text{ISE}), 2)$ & -1.7562 & 0.7361 & -2.386 & 0.018 * \\
\bottomrule
\end{tabular}
\end{table}

\textbf{Bondad de Ajuste:}
\begin{itemize}
    \item $R^2 = 0.9918$ (explica 99.18\% de varianza)
    \item $R^2_{\text{adj}} = 0.9913$
    \item F-statistic = 2329 (p $<$ 0.001)
    \item SE residual = 0.2412 pp
    \item Observaciones = 227
\end{itemize}

\subsection{Interpretación de Coeficientes}

\subsubsection{Política Monetaria (DTF)}

\textbf{Efecto Contemporáneo:}
Un aumento de 1 punto porcentual en DTF reduce la inflación en 0.2307 puntos 
porcentuales (23.07 pb por 100 pb), con significancia estadística al nivel 
1\% (p = 0.0025).

\textbf{Efectos Rezagados:}
\begin{itemize}
    \item Lag 1: -0.0925 pp (no significativo)
    \item Lag 2: -0.1260 pp (marginalmente significativo, p = 0.081)
\end{itemize}

\textbf{Suma Total (2 meses):} 0.0123 pp

\textbf{Multiplicador de Largo Plazo:}
\[
MP_{LP}^{\text{DTF}} = \frac{0.0123}{0.0375} = 0.328
\]

Interpretación: Un aumento permanente de 100 pb en DTF reduce la inflación en 
32.79 pb a largo plazo.

\subsubsection{Pass-Through Cambiario (TRM)}

\textbf{Efecto Máximo (Lag 1):}
Un aumento de 1\% en la depreciación anual del TRM aumenta la inflación en 
1.0524 pp el mes siguiente (105.24 pb), marginalmente significativo 
(p = 0.069).

\textbf{Reversión Parcial (Lag 2):}
-0.8435 pp (significativo al 5\%, p = 0.031)

\textbf{Pass-Through de Largo Plazo:}
\[
PT_{LP} = \frac{0.4154}{0.0375} = 11.07
\]

Interpretación: Depreciación permanente de 1\% aumenta inflación LP en 11.07 
pp (pass-through de 1107\%, amplificado por persistencia).

\subsubsection{Actividad Económica (ISE)}

\textbf{Efecto Máximo (Lag 1):}
Un aumento de 1\% en ISE aumenta la inflación en 3.7858 pp el mes siguiente 
(378.58 pb), \textbf{altamente significativo} (p $<$ 0.0001).

\textbf{Interpretación Económica:}
Fuerte Curva de Phillips: expansiones económicas presionan inflación mediante 
presión de demanda agregada.

\subsection{Diagnósticos del Modelo}

\begin{table}[h]
\centering
\caption{Pruebas de Diagnóstico}
\label{tab:diag}
\begin{tabular}{lrrl}
\toprule
Test & Estadístico & P-valor & Conclusión \\
\midrule
Breusch-Godfrey (AC lag 1) & 1.42 & 0.233 & [OK] No hay AC \\
Ljung-Box (AC lag 12) & 41.98 & $<$0.001 & [ALERTA] AC en lags altos \\
Breusch-Pagan (Hetero) & 19.29 & 0.056 & [OK] Homocedasticidad \\
Shapiro-Wilk (Normalidad) & 0.991 & 0.151 & [OK] Normal \\
\bottomrule
\end{tabular}
\end{table}

\textbf{Evaluación:}
\begin{itemize}
    \item [OK] Modelo es fundamentalmente sólido
    \item [ALERTA] Autocorrelación en lags superiores sugiere variable omitida
    \item Recomendación: Usar SE robustos (HC3/HC4)
\end{itemize}

% =============================================================================
\section{Conclusiones}
% =============================================================================

Este análisis proporciona \textbf{evidencia robusta} de que la política 
monetaria es un \textbf{instrumento efectivo} para el control de la inflación 
en Colombia.

\subsection{Hallazgos Principales}

\begin{enumerate}
    \item \textbf{DTF reduce inflación:} Aumento de 100 pb en DTF reduce 
          inflación en 23-33 pb (CP-LP), estadísticamente significativo 
          (p = 0.0025).
    
    \item \textbf{Canales operativos:} Los tres canales teóricos están 
          activos:
          \begin{itemize}
              \item Demanda agregada (ISE muy significativo)
              \item Pass-through cambiario (TRM significativo en lag 1)
              \item Expectativas (alta persistencia AR = 96\%)
          \end{itemize}
    
    \item \textbf{Velocidad de transmisión:} Efectos principales en 0-2 meses, 
          acumulación hasta 12 meses.
    
    \item \textbf{Persistencia inflacionaria:} Muy alta (suma AR = 0.9625), 
          implica que shocks tardan muchos meses en desaparecer.
\end{enumerate}

\subsection{Limitaciones}

\begin{enumerate}
    \item \textbf{Causalidad no probada:} DTF puede ser endógena a inflación. Solución: Test Granger causality.
    
    \item \textbf{Cambios estructurales:} Período incluye 2008 (crisis), 2015 
(petróleo), 2020 (COVID). Solución: Chow test.
    
    \item \textbf{Autocorrelación residual:} En lags altos. Solución: SE 
          robustos HC3/HC4.
\end{enumerate}

\subsection{Recomendaciones Futuras}

\begin{enumerate}
    \item Validación: Granger causality, Chow test, rolling regressions
    \item Extensiones: VECM si cointegración, VAR estructural
    \item Mejoras: Incluir brecha de producto, expectativas de inflación
\end{enumerate}

% =============================================================================
\appendix
% =============================================================================

\section{Código R}

El análisis completo se reproduce con:

\begin{lstlisting}
cd ~/EAFIT/2026-1/Econometría\ 2/Retos/Reto1_PoliticaMonetaria_Colombia
Rscript scripts/04_ADL_SIMPLIFICADO.R
\end{lstlisting}

Outputs:
\begin{itemize}
    \item \texttt{outputs/ADL/modelo\_ADL.rds} — Modelo estimado
    \item \texttt{outputs/ADL/datos\_adl.rds} — Dataset transformado
    \item \texttt{outputs/ADL/06\_diagnosticos\_ADL.pdf} — Gráficos
\end{itemize}

\section{Referencias}

\begin{thebibliography}{99}

\bibitem{bernanke1992} Bernanke, B.S., \& Blinder, A.S. (1992). ``The Federal 
Funds Rate and the Transmission of Monetary Policy.'' \textit{American Economic 
Review}, 82(4), 901-921.

\bibitem{fuhrer2009} Fuhrer, J. (2009). ``The Slow Pass-Through Puzzle: 
Incomplete Nominal Adjustment in the Recent Inflation Process.'' 
\textit{Federal Reserve Bank of Boston Public Policy Briefs}, 9-2.

\bibitem{pesaran1998} Pesaran, M.H., \& Shin, Y. (1998). ``An Autoregressive 
Distributed-Lag Modelling Approach to Cointegration Analysis.'' 
\textit{Journal of Econometrics}, 94(1-2), 31-52.

\bibitem{romer1989} Romer, C.D., \& Romer, D.H. (1989). ``Does Monetary Policy 
Matter? A New Test in the Spirit of Friedman and Schwartz.'' NBER Working 
Paper No. 2966.

\bibitem{taylor1999} Taylor, J.B. (1999). \textit{Monetary Policy Rules}. 
University of Chicago Press.

\bibitem{br2019} Banco de la República. (2019). ``Mecanismos de Transmisión de 
la Política Monetaria en Colombia.'' 

\bibitem{wooldridge2015} Wooldridge, J.M. (2015). \textit{Introductory 
Econometrics: A Modern Approach} (5th ed.). South-Western.

\end{thebibliography}

\end{document}
